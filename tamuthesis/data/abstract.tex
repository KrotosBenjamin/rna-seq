%%%%%%%%%%%%%%%%%%%%%%%%%%%%%%%%%%%%%%%%%%%%%%%%%%%
%
%  New template code for TAMU Theses and Dissertations starting Fall 2012.  
%  For more info about this template or the 
%  TAMU LaTeX User's Group, see http://www.howdy.me/.
%
%  Author: Wendy Lynn Turner 
%	 Version 1.0 
%  Last updated 8/5/2012
%
%%%%%%%%%%%%%%%%%%%%%%%%%%%%%%%%%%%%%%%%%%%%%%%%%%%
%%%%%%%%%%%%%%%%%%%%%%%%%%%%%%%%%%%%%%%%%%%%%%%%%%%%%%%%%%%%%%%%%%%%%
%%                           ABSTRACT 
%%%%%%%%%%%%%%%%%%%%%%%%%%%%%%%%%%%%%%%%%%%%%%%%%%%%%%%%%%%%%%%%%%%%%

\chapter*{ABSTRACT}
\addcontentsline{toc}{chapter}{ABSTRACT} % Needs to be set to part, so the TOC doesnt add 'CHAPTER ' prefix in the TOC.

\pagestyle{plain} % No headers, just page numbers
\pagenumbering{roman} % Roman numerals
\setcounter{page}{2}

\indent Human chromosome 15q11-q13 contains a cluster of imprinted genes that are associated with a number of neurological disorders that exhibit non-Mendelian patterns of inheritance, such as Angelman syndrome (AS) and Prader-Willi syndrome. Angelman syndrome is caused by the loss-of-expression of maternally inherited ubiquitin E3A protein ligase gene (\textit{UBE3A}). Prader-Willi syndrome is caused by loss-of-function of paternally inherited \textit{SNORD116} snoRNAs (small nucleolar RNAs), which are expressed as part of a long polycistronic transcriptional unit (PTU) comprised of \textit{SNURF-SNRPN}, additional orphan C/D box snoRNA clusters, and the \textit{UBE3A} antisense transcript (\textit{UBE3A-AS}). The full-length transcript of PTU, including \textit{UBE3A-AS}, is only expressed in neurons causing the imprinting of paternal \textit{UBE3A}. Why this occurs in only neurons remains largely unknown. Furthermore, this neuron-specific imprinting adds additional difficulty for therapeutic intervention. In this dissertation, the imprinting mechanism of \textit{UBE3A} is examined in detail, while an alternative high-throughput screening (HTS) method for drug discovery in neurons is developed.

A combination of bioinformatic and molecular analysis of the human and mouse PTU revealed that \textit{UBE3A-AS/Ube3a-AS} is extensively processed via 5' capping, 3' polyadenylation and alternative splicing, suggesting that the antisense may have regulatory functions apart from imprinting \textit{UBE3A} in neurons. Following this discovery, the transcripional profiles and processing of mouse paternal \textit{Ube3a} was investigated as literature suggested that imprinted paternal \textit{Ube3a}, unlike other imprinted genes, was transcribed up to intron 4. This analysis unveiled a fourth \textit{Ube3a} isoform that terminates within intron 4. Moreover, expression of this isoform correlated with \textit{Ube3a-AS} expression, suggesting alternative reasons for the imprinting of \textit{Ube3a}. In addition to the analysis of the imprinting of \textit{Ube3a}, an alternative solution for drug discovery for central nervous system disorders was developed and validated. Here, an embryonic stem cell-derived neuronal culture system was developed for HTS and tested using the paternal \textit{Ube3a$^{YFP}$} reporter cell-line. Using a known reactivator of paternal \textit{Ube3a}, Topotecan - a topoisomerase inhibitor, as a positive control a proof-of-concept study demonstrated the utility of this method for HTS drug discovery. Collectively, these results advance the field and understanding of antisense lncRNAs and provide a versatile tool for drug discovery for neurological disorders.

\pagebreak{}
