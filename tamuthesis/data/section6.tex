%%%%%%%%%%%%%%%%%%%%%%%%%%%%%%%%%%%%%%%%%%%%%%%%%%%%%%%%%%%%%%%%%%%%%%
%%                           SECTION VI
%%%%%%%%%%%%%%%%%%%%%%%%%%%%%%%%%%%%%%%%%%%%%%%%%%%%%%%%%%%%%%%%%%%%%

\chapter{\uppercase {Conclusion}}

Recent studies show that the human brain has one of the most complex expression patterns of the body. The expression patterns of the imprinted region 15q11-13, containing \textit{UBE3A} and \textit{UBE3A-AS}, are no exception. The studies herein explore the unique imprinting of paternal \textit{UBE3A/Ube3a} in neurons. Specifically, we were able to demonstrate that both mouse and human \textit{Ube3a-AS/UBE3A-AS} are extensively processed via bioinformatics and molecular analysis; demonstrating spatiotemporal regulation of \textit{Ube3a-AS}. For both mouse and humans, we sequenced 3' RACE clones verifying polyadenylation sites within the antisense region of \textit{Ube3a/UBE3A}. This finding suggesting that dosage sensitive \textit{UBE3A} may be imprinted in neurons so that sense and antisense transcripts are co-expressed in the brain. Additionally, the temporal regulation of \textit{Ube3a-AS} suggest it may play a role in the developing brain. Following these experiments, we demonstrated the existence of a fourth paternal-specific \textit{Ube3a} isoform in mice generated via alternative polyadenylation at novel exon 4.1. In the investigating of this novel isoform, we demonstrated that its expression corresponded with \textit{Ube3a-AS} expression suggesting an alternative mechanism of paternal \textit{Ube3a} imprinting. Simultaneously, we developed a high-throughput screening method in embryonic stem cell-derived neurons; demonstrating its potential for drug discovery for neurodevelopmental disorders like Angelman syndrome. Taken together, these results suggest the lncRNA, \textit{UBE3A-AS}, may have additional regulatory functions outside of the imprinting of \textit{UBE3A} and that the novel isoform 4 may also serve as a regulatory RNA. Moreover, they also propose an alternative mechanism of imprinting for \textit{Ube3a}, wherein by some unknown mechanism \textit{Ube3a-AS} causes termination and polyadenylation of paternal \textit{Ube3a} transcription. Finally, these results setup future drug discovery experiments for Angelman syndrome.

\section{\textit{Ube3a-AS} demonstrates complex expression within the brain}

The antisense transcript of \textit{UBE3A}, identified in 1998, originates from the 3' end of a large polycistronic transcriptional unit consisting of \textit{SNURF/SNRPN}, \textit{IPW}, and a cluster of snoRNAs \cite{Rougeulle1998,Landers2004,Runte2004,Runte2001}; however, \textit{UBE3A-AS} is only expressed in neurons. While \textit{Ube3a-AS} is sufficient to repress the expression of \textit{Ube3a} in neurons \cite{Meng2012,Meng2013}, our findings showed temporal regulation only for \textit{Ube3a-AS} and \textit{Ube3a} isoform 4, suggesting \textit{Ube3a-AS} expression is unrelated to the imprinting of \textit{Ube3a} and therefore may be a by-product of another regulatory process. In fact, additional work in our laboratory showed that maternal Ube3a expression increased as paternal expression decreased keeping overall Ube3a expression constant \cite{Hillman2017}. Further investigation into \textit{UBE3A-AS/Ube3a-AS} expression revealed that the antisense transcript was alternatively spliced generating dozens of transcripts in the mouse and at least ten transcripts in the human with several polyadenylation sites and 5' capping events. In the mouse, we confirmed RNA-seq analysis that these isoforms were differentially expressed between brain regions. Furthermore, we found that \textit{Ube3a-AS} generalized isoforms 1 and 2 along with \textit{Ube3a} isoform 4 were temporally regulated in the hippocampus, showing upregulation during post-natal development.

These findings contradict current \textit{Ube3a-AS} theories  mainly because of differing approaches in analyzing the 15q11-13 imprinted region. Here, we used powerful high-throughput sequencing technologies, specifically stranded paired-end RNA-seq, to generate transcriptional profiles where sense and antisense transcripts could be distinguished. Additionally, we used the hybrid C57xDBA mice from public data to further distinguish allelic contributions. As such, we were able to analyze the imprinted region in a more detailed and rigorous manner than other studies, and in doing so our findings showed spatiotemporally regulation of \textit{Ube3a-AS}.

Spatiotemporal regulation used to generated specialized tissues during development in a highly dynamic environment like the brain. For instance, genes associated with temporal regulation during early embryonic development are grouped into categories involving neuron differentiation, axonogensis, and forebrain development, which are often connected to proper morphological growth \cite{Liscovitch2013}. Furthermore, temporally regulated genes associated with late post-natal development involve the regulation of synaptic transmission, behavior, and learning and memory \cite{Liscovitch2013}. 
As such, the upregulation of \textit{Ube3a-AS} during post-natal development, is in agreement with the imprinting of \textit{Ube3a} and the phenotypes associated with neurodevelopmental disorders of the imprinted 15q11-q13 region. Moreover, the uniqueness of cerebellum temporal gene expression is also reflected in our results with differential expression of the isoforms within the brain regions \cite{Liscovitch2013,Lein2006,Zapala2005}. As such, it is also possible that the generalized \textit{Ube3a-AS} isoform 3 could be temporally regulated in the cerebellum, where its expression is upregulated. Altogether, these results suggest a regulatory function for \textit{UBE3A-AS}.

The implications of \textit{UBE3A-AS} having additional functions potentially impacts the current therapeutic strategy for Angelman syndrome. Since the inactive paternal \textit{UBE3A} allele is genetically intact but epigenetically silent, recent studies have targeted the paternal \textit{Ube3a} for reactivation via disruption of \textit{Ube3a-AS} \cite{Meng2013,Elgersma2007,Huang2012,Meng2015,Shi2015,Bailus2016}; however, if \textit{UBE3A-AS} has additional regulatory functions, disruption of the lncRNA could adversely effect other pathways that \textit{UBE3A-AS} may play a role in. It is for this reason, that \textit{UBE3A-AS} should be intensively studied to better understand any potential ramification of disrupting its expression as a treatment for Angelman syndrome.

\section{\textit{Ube3a-AS} generates a paternal, neuron-specific, \textit{Ube3a} isoform - isoform 4}

Previous investigations into the expression of \textit{Ube3a-AS} revealed partial expression of paternal \textit{Ube3a} \cite{Meng2012,Numata2011}. This phenomenon along with ChIP-on-chip experiments confirming that both \textit{Ube3a} promoters are enriched with chromatin modification (histone H3 lysine 4 trimethylation), bound by RNA polymerase II, and actively transcribed gave rise to the collision model as a mechanism of paternal \textit{Ube3a} \cite{Meng2013} imprinting in neurons. In this model, RNA polymerase II from \textit{Ube3a} sense-strand and \textit{Ube3a-AS} antisense-strand collide causing a gradual decrease in expression resulting in partial paternal \textit{Ube3a} expression, \textbf{Figure \ref{Figure 1-12: }A}. As noted in \textbf{Chapter 1}, this model implies that \textit{Ube3a-AS} is not expressed past the collision within intron 4 of \textit{Ube3a}, which is not in agreement with Numata \textit{et al.} SNP data showing paternal RNA expression upstream the \textit{Ube3a} primer \cite{Numata2011}. From our \textit{Ube3a-AS} isoform annotations (\textbf{Figure \ref{genome annotation}}) several assembled transcripts aligned with Numata \textit{et al.} SNP data upstream the \textit{Ube3a} promoter, suggesting that \textit{Ube3a-AS} RNA polymerase does not stall within \textit{Ube3a} gene. Furthermore, we identified and verified a fourth \textit{Ube3a} isoform that terminates within intron 4 in agreement with previous studies \cite{Meng2012,Numata2011}. Altogether suggesting an alternative model of imprinting \textit{alternative polyadenylation}, wherein \textit{Ube3a-AS} though some unknown mechanism leads to alternative exon usage, exon 4.1, and termination of \textit{Ube3a} transcription.

Alternative polyadenylation (APA) is a widespread phenomenon within humans that generates isoforms with alternative 3' ends. Moreover, APA events occur commonly during development and cellular differentiation in a tissue-specific manner \cite{Zhang2005}. For example, BDNF, or brain-derived neurotrophic factor, is one such brain-specific sense/antisense gene that involves APA to produce short and long 3' UTR isoforms \cite{Timmusk1993,Lau2010}. Although unlike the proposed \textit{Ube3a/Ube3a-AS} model, BDNF APA sites are within the 3' UTR. Even so, this type of alternative polyadenylation, terminal exon APA usage hypothesized in \textbf{Figure \ref{Figure 1-12: }B} is known to associated with splicing and APA machinery \cite{Elkon2013}. In the \textit{Ube3a/Ube3a-AS} example, a weak 5' splicing site within the long intron 4 leads to dynamic competition between splicing and polyadenylation \cite{Wang2008}, which causes the alternative polyadenylation within intron 4 - paternal \textit{Ube3a} isoform 4. 

With the understanding that tissue-specific APA is not a unique phenomenon, we proposed three possible reasons for the imprinting of \textit{Ube3a} in neurons. The first involves the importance of paternal \textit{Ube3a} isoform 4 caused by APA. Here, isoform 4, like many APA isoforms, could change microRNA binding sites \cite{Sandberg2008}, or potentially code for a new protein \cite{Alt1980,Tian2007}. Additionally, changes in 3' UTR have effects to stability, cellular localization, and translation efficiency \cite{Fabian2010,Andreassi2009}, all of which effect RNA functionality. The second reason could be that \textit{Ube3a-AS} has tissue-specific regulatory functions and the imprint evolved to express both \textit{Ube3a} and \textit{Ube3a-AS}. Finally, it is possible that \textit{Ube3a-AS} and \textit{Ube3a} isoform 4 expression have regulatory roles in brain development. Nevertheless, there are clear ramifications for current Angelman syndrome drug therapy strategies as discussed above.

Even so, the revelation of alternative polyadenylation generating paternal \textit{Ube3a} isoform 4 opens the door for an alternative approach to AS drug therapies like exon skipping. Exon skipping is a form of RNA splicing used to restore reading frame within a gene and can be experimentally done via antisense-mediation. Antisense-mediation uses antisense oligonucleotides to hybridize to a sense target sequence leading to RNase H cleaving and gene-specific knockdown \cite{Hausen1970}. For exon skipping purposes, the antisense oligonucleotides are modified to act on pre-mRNA splicing by blocking splicing signals and induces exon skipping \cite{Kole2001}. Currently, this technique is being used in phase III clinical trials held by Sarepta Therapeutics for Duchenne muscular dystrophy (DMD) \cite{Kole2015}. As DMD is often caused by mutation that disrupt the open reading frame, exon skipping is used to modulate splicing of \textit{DMD} gene to restore the reading frame \cite{Hoffman1987,Monaco1988,Koenig1989}. Since \textit{Ube3a} isoform 4 uses an alternative exon, exon 4.1, that terminates transcription, antisense oligonucleotides can be designed to skip exon 4.1 and possible produce a full-length paternal \textit{Ube3a} transcript.

\section{Embryonic stem cells are a versatile source for high-throughput screening}

In \textbf{Chapters 4 \textit{and} 5}, embryonic stem cells were used to generate neurons ultimately for HTS for Angelman syndrome. These ES cells were relatively easy to generate and expand for large-scale studies unlike primary neurons, which have a finite number of cells per animal. Additionally, ES cells from \textit{Ube3a$^{+/YFP}$} mice housed in the laboratory were produced within months demonstrating the modularity of the HTS method. While ES cell-derived neuronal cultures generate significantly more cells than primary cultures, there are other sources of immortalized neuronal cell-lines (e.g., P19, SH-SY5Y neuroblastoma cells, NT2, PC12 cells, etc.) that could potentially do the same; however, iPS (induced pluripotent stem) cells are the only immortalized cell-line surging in popularity for HTS purposes \cite{Chambers2012,Crompton2013,Engle2014}.

One of the major benefits of using human iPS cells over ES cells (human or mouse) is the ability to use patient-specific cells. As these cells are derived from postnatal somatic cells \cite{Takahashi2007b,Yu2007}, they can be collected directly from diseased individuals for more biological complex drug screenings. While the advantages of using iPS cells over ES cell appear abundantly clear, especially for translational medicine applications, there are still several concerns with the use of iPS cells including functionality and chromosomal aberrations \& genetic modifications. For example, recent studies have shown variability in iPS neurons differentiation efficiency compared to more consistent differentiation using ES cells \cite{Hu2010}. Moreover, high-resolution genetic and epigenetic analysis revealed differences between iPS and ES cells including DNA methylation and expression profiles \cite{Chin2009,Deng2009}, which could effect the ability of iPS cells to recapitulate certain diseases accurately. Although it is not clear what these results mean, more research is required to understand which stem cell-line is better suited for HTS.

Moving forward with HTS in ES cell-derived neurons, three key recommendations are provided.
\textbf{1)} Test multiply lots of NeuN antibody before purchasing in bulk ($> 20$ vials).\footnote{NeuN antibody varied greatly depending on lot number}
\textbf{2)} Run no more than 12-13 plates in one setting.\footnote{Assuming image acquisition is approximate 90 min/plate, otherwise, plate number should be adjusted accordingly.}
\textbf{3)} Always add positive (Topotecan) and negative (Vehicle) controls to each plate for quality control.

\section{Future studies}
\subsection{Investigation of \textit{Ube3a-AS/UBE3A-AS} predicted transcripts}
\subsubsection{PacBio}
While we were able to several polyadenylation sites of \textit{UBE3A-AS/Ube3a-AS}, the majority of the predicted transcripts have yet to be identified. Moreover, we were unable to verify full-length transcripts with the short-read RNA-seq. To circumvent short-read problems for isoform sequencing, full-length cDNA sequences can be generated using PacBio long-read technologies to determine the exact number of isoforms, the degree of interconnection between upstream snoRNAs, and exon usage between isoforms. 

\subsubsection{circRNAs}
In both human and mouse assemblies of the antisense region, transcripts that appeared to be circular RNAs (circRNAs) visually were detected. These transcripts showed significant brain-specific expression compared to the generalized isoform categories for human and mouse. First, bioinformatic algorithms like STAR circRNA function \cite{Dobin2012} or by following Memczak \textit{et al.} protocol \cite{Memczak2013} to predict circRNA and cross-referenced with the databases of annotated circRNAs \cite{Guo2014}. Their expression can then be verified with qPCR techniques as described in Li \textit{et al.} (2017) \cite{Li2017}. Briefly, total RNA would be digested with RNase R and purified with phenol-chloroform extraction to be used in cDNA reaction and subsequent circRNA specific qPCR assays \cite{Memczak2013}. Results from these experiments would provide additional support for regulatory functions of \textit{UBE3A-AS/Ube3a-AS}. 

\subsection{Additional verification of \textit{Ube3a-AS} control of \textit{Ube3a} isoform 4 expression}
As Topotecan is known to effect alternative splicing \cite{Shkreta2008,Powell2013b}, additional experiments that are \textit{Ube3a-AS} specific could also be conducted. One means of doing this is by using antisense oligonucleotides (ASOs) like Meng \textit{et al.} (2013) \cite{Meng2013}. In this approach, ASOs from Meng \textit{et al.} (2013) paper can be used to treated primary neuronal cultures and qPCR can be conducted similarly to Topotecan treated neurons in \textbf{Chapter 3}. In addition to ASOs, a PWS-IC deletion transgenetic mouse model could be used. For this approach, RNA would extracted from transgenetic mouse brains and expression could be measured with RT-PCR. Results from these experiments would provide additional support for the control of alternative polyadenylation of \textit{Ube3a} isoform 4 by \textit{Ube3a-AS}.

\subsection{Molecular analysis of \textit{Ube3a-AS} and \textit{Ube3a} isoform 4}
\subsubsection{RNA stability analysis}
Recent studies investigated the processing of \textit{Ube3a-AS} determined that \textit{Ube3a-AS} was an atypical RNAPII transcript \cite{Meng2012}; however, we found that the antisense primers listed in the publication targeted introns of the spliced \textit{Ube3a-AS} predicted by our RNA-seq analysis. As such, the processing of \textit{Ube3a-AS} should be re-evaluated for polyadenylation, cellular localization, and stability.
To test polyadenylation, processed mRNA can be compared to total RNA with quantitative PCR similar to Meng \textit{et al.} (2012) \cite{Meng2012}. To test localization of sense and antisense transcript, RNA FISH (fluorescence \textit{in situ} hybridization) can be used on generalized \textit{Ube3a-AS} isoforms and \textit{Ube3a} isoform 4. This would allow for visualization of these transcripts at the sub-cellular level in a quantifiable manner. Stellaris offers several protocols for performing RNA FISH on their website (\url{https://www.biosearchtech.com/support/resources/stellaris-protocols}).
Finally, RNA stability can be determined via half-life experiments on actinomycin D treated primary neuronal cultures as previously described \cite{Meng2012}.

\subsubsection{Temporal regulation of the antisense transcripts}
The temporal regulation analysis was limited to mouse hippocampal neurons, as such, addition molecular experiments could be conducted to look at temporal regulation in mouse cerebellum with qPCR techniques. Furthermore, temporal regulation in the human brain can also be investigated with publicly available RNA-seq data \cite{Liu2016}. Both analysis could be used in conjunction with the Allen Brain Atlas to determine possible pathways for the \textit{Ube3a-AS/UBE3A-AS}.

\subsubsection{RNA Co-Immunoprecipitation}
In addition to the temporal expression analysis, RNA co-immunoprecipitation to elucidate possible RNA-protein interactions for \textit{Ube3a-AS} and \textit{Ube3a} isoform 4. Jedamzik and Eckmann (2009) provide a full protocol for analyzing RNA-protein complexes by RNA co-immunoprecipitation \cite{Jedamzik2009}. Results from this study could be also be used in pathway analysis.

\subsection{Exon skipping of \textit{Ube3a} isoform 4 as a therapeutic strategy for Angelman syndrome}

As previously mentioned, the identification of a paternal-specific \textit{Ube3a} isoform generated from alternative polyadenylation offers the potential of an alternative therapeutic strategy for Angelman syndrome, exon skipping. While not a trivial strategy, there is a comprehensive protocol book for exon skipping \cite{Aartsma-Rus2012}. In brief, potential splice sites and cis-regulatory elements (i.e. exonic and intronic splicing enhancer sequences) would first be identified via open-source software like ESEfinder 3.0 \cite{Smith2006,Cartegni2003}. Following identification, a splicing functional assay based on minigenes would be used, where the exonic fragment surround \textit{Ube3a} exon 4.1 is amplified and cloned into a splicing competent minigene vector. If successful, the antisense oligonucleotides used in the \textit{in vitro} splicing assay can be optimized for animal models. Furthermore, the protocol book edited by Aartsma-Rus has several protocols to help with troubleshooting and optimization of exon skipping.

\subsection{High-throughput screening assays}

\textbf{Chapters 4 \textit{and} 5} developed and verified the ES cell-derived neuronal culture HTS method. With this frame work in place, the next step is to perform a small molecule drug screen. Following a small molecule screen, hits would be verified via a primary neuron assay and drug dosage profile created for each positive hits. After which, animals can be used to further verify results. In addition to this small molecule screening, an RNA-interference HTS can also be run. Despite numerous studies into \textit{UBE3A}, little is known about its pathway in neurons. An RNA-interference screen could give some insight into the pathways involved with the imprint of \textit{Ube3a} in neurons.
